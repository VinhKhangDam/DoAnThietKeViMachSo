\documentclass[a4paper,12pt]{article}
\usepackage[utf8]{inputenc}
\usepackage[vietnamese]{babel}
\usepackage{titlesec}
\usepackage{geometry}
\usepackage{graphicx}
\usepackage{fancyhdr}
\usepackage{tocloft}
\usepackage{hyperref}
\usepackage{setspace}
\usepackage{amsmath}
\usepackage{float}
\usepackage[most]{tcolorbox}
\usepackage{tikz}
\usepackage{everypage-1x}
\usepackage{atbegshi}
\usepackage{eso-pic}
\usepackage{enumitem}

\geometry{left=2cm,right=1.5cm,top=2cm,bottom=2.5cm}
\onehalfspacing

\setcounter{secnumdepth}{4}
\setcounter{tocdepth}{4}
\renewcommand{\theparagraph}{\thesubsubsection.\arabic{paragraph}}
\titleformat{\paragraph}[block]{\normalfont\normalsize\bfseries}{\theparagraph}{1em}{}
\titleformat{\section}{\bfseries\large}{\thesection}{1em}{}
\titleformat{\subsection}{\bfseries\normalsize}{\thesubsection}{1em}{}
\renewcommand{\cftsecleader}{\cftdotfill{\cftdotsep}}
\renewcommand{\cftsecfont}{\normalfont}
\renewcommand{\cftsecpagefont}{\normalfont}
\renewcommand{\cfttoctitlefont}{\normalfont\bfseries\Large}
\renewcommand{\cftaftertoctitle}{\hfill}
\renewcommand{\thesection}{\Roman{section}} % Section: La Mã
\renewcommand{\thesubsection}{\arabic{subsection}} % Subsection: 1, 2, 3...
\renewcommand{\thesubsubsection}{\arabic{subsection}.\arabic{subsubsection}} % Subsubsection: 1.1, 2.1,...
% Vẽ khung cho trang bìa
\usepackage{atbegshi}
\AddToShipoutPictureBG*{%
    \ifnum\value{page}=1
    \AtPageLowerLeft{%
        \begin{tikzpicture}[remember picture, overlay]
            \draw[thick, black] 
                ([xshift=1.5cm, yshift=-1.5cm] current page.north west)
                rectangle 
                ([xshift=-1.5cm, yshift=2cm] current page.south east);
        \end{tikzpicture}
    }
}

\hypersetup{
    colorlinks=true, % Bỏ khung đỏ, chỉ hiển thị liên kết màu
    linkcolor=black, % Màu liên kết trong nội dung
    filecolor=black, % Màu liên kết file
    urlcolor=blue, % Màu liên kết URL
    citecolor=black % Màu liên kết trích dẫn
}


\begin{document}
%----------------------------------
% Trang bìa
%----------------------------------
\thispagestyle{empty}
\begin{center}
\textbf{\Large ĐẠI HỌC QUỐC GIA THÀNH PHỐ HỒ CHÍ MINH}\\
\textbf{\Large TRƯỜNG ĐẠI HỌC CÔNG NGHỆ THÔNG TIN}\\
\textbf{\Large KHOA KỸ THUẬT MÁY TÍNH}\\[1cm]

\begin{figure}[H]
    \centering
    \includegraphics[width=0.3\textwidth]{../PNG/CE.png}
    \label{fig:LOGO_CE}\\
\end{figure}

\vspace {1cm}

\textbf{\Large BÁO CÁO ĐỒ ÁN THIẾT KẾ VI MẠCH SỐ}\\[0.5cm]
\textbf{\Large STICK DIAGRAM}\\[5cm]

\begin{flushleft}

\textbf{Lớp:} CE222.P21\\
\textbf{Giảng viên hướng dẫn:} ThS. Ngô Hiếu Trường\\
\textbf{Sinh viên thực hiện:} \\
- Đàm Vĩnh Khang : 22520606 \\
- Nguyễn Tuấn Khoa : 22520681\\[3.5cm]

\end{flushleft}
\end{center}
\begin{center}
\textbf{Thành phố Hồ Chí Minh, tháng 5 năm 2025}
\end{center}
\newpage
\setcounter{page}{1}
%----------------------------------
% Mục lục
%----------------------------------
\tableofcontents
\newpage

%----------------------------------
% Nội dung
%----------------------------------

%I.
\section{Tổng quát}
\begin{center}
\textbf{\large Bài tập lớn môn thiết kế vi mạch số CE222}
\end{center}
• \textbf{Chủ đề:} Chuyển đổi biểu thức Boolean thành Stick Diagram.\\
• \textbf{Ngôn ngữ lập trình được sử dụng:} C++ và Python\\
• \textbf{Input:} Biểu thức Boolean đối xứng

Ví dụ: \( Y = \overline{(A + B + C) * D} \) thì input là: \( (A + B + C) * D \).\\
• \textbf{Output:} Stick Diagram\\
• \textbf{Điều kiện Input để chương trình hoạt động đúng:}

    + Biểu thức Boolean phải rõ ràng và tối ưu vì chương trình chưa có khả năng tối ưu biểu thức

    + Không giới hạn số biến, tuy nhiên phải lớn hơn 1 và phù hợp nhất là 3 tới 5 biến
\newpage
%II.
\section{Quy trình thực hiện}
Quy trình thực hiện gồm 3 giai đoạn:

1 . Tìm hiểu thuật toán và ngôn ngữ lập trình

2. Hiện thực hóa phần xử lí bằng C++ và vẽ Stick Diagram bằng Python

3. Mô phỏng và viết báo cáo\\
Phân công công việc:

• Vĩnh Khang: Mô hình hóa schematic diagram và tìm đường đi Euler trên C++, viết báo cáo.

• Tuấn Khoa: Vẽ Stick Diagram bằng Python và viết báo cáo.
%1.
\subsection{Tìm hiểu thuật toán}
Để mô phỏng Stick Diagram từ biểu thức Boolean, thì cần phải có schematic diagram để biểu diễn biểu thức Boolean gồm: \\
• PMOS pull-up network\\
• NMOS pull-down network\\
• Input, output, GND và VDD\\
Ví dụ: \( Y = \overline{A *(B + C) + D * E} \)\\
\begin{figure}[H]
    \centering
    \includegraphics[width=0.5\textwidth]{../PNG/ViDuSchematic.jpg}
    \label{fig:Ex_Schematic}\\
\end{figure}

Tiếp theo, chúng ta cần tìm đường đi Euler của schematic diagram cho cùng NMOS pull-down và PMOS pull-up.\\
Ở đây, chúng ta sẽ tìm đường đi Euler cho vùng PMOS pull-up đầu tiên và sau đó cho vùng NMOS pull-down sẽ giống với PMOS tìm được.\\

\begin{figure}[H]
    \centering
    \includegraphics[width=0.5\textwidth]{../PNG/Euler_CMOS.jpg}
    \label{fig:Ex_Schematic}\\
\end{figure}

\begin{figure}[H]
    \centering
    \includegraphics[width=0.4\textwidth]{../PNG/Euler_alone.jpg}
    \label{fig:Ex_Schematic}\\
\end{figure}

Cuối cùng, ta sẽ vẽ Stick Diagram từ đường đi Euler đã tìm được.\\

\begin{figure}[H]
    \centering
    \includegraphics[width=0.5\textwidth]{../PNG/Stick_handwrite.jpg}
    \label{fig:Ex_Schematic}\\
\end{figure}

Như vậy, thuật toán để biểu diễn stick diagram từ biểu thức Boolean là:\\
\textbf{Vẽ schematic diagram -> Tìm đường đi Euler -> Vẽ stick diagram}.
\newpage
%2.
\subsection{Hiện thực bằng Python và C++}
%2.1
\subsubsection{Vẽ Schematic và tìm đường đi Euler trên C++}
Đây là phần khó nhất của bài tập, vì từ một biểu thức Boolean, ta có thể vẽ được rất nhiều trường hợp của stickdiagram.
Nhưng đối với yêu cầu của bài toán, ta phải vẽ được trường hợp của stick diagram của biểu thức Boolean được nhập từ bàn phím
sao cho vùng NMOS pull-down và PMOS pull-up có chung đường đi Euler.
%2.1.1
\paragraph{Hướng tiếp cận}
Không thể vẽ schematic diagram 1 cách trực tiếp từ biểu thức Boolean.
Thay vào đó, ta sẽ sử dụng đồ thị (Graph) để mô hình hóa schematic diagram.\\
Vì schematic diagram gồm 2 vùng là NMOS pull-down network và PMOS pull-up network, nên ta sẽ tạo 2 đồ thị tương ứng với 2 vùng này, 
với node là các biến đầu vào của Boolean.\\
Ví dụ: \( Y = \overline{A *(B + C) + D * E} \) thì Node sẽ là:
\textbf{A, B, C, D, E}\\
%2.1.2
\paragraph{Tạo đồ thị mô hình hóa NMOS pull-down network và tìm đường đi Euler}
%a. và b.
\begin{enumerate}[label=\alph*.]
    \item{Tạo đồ thị mô hình hóa NMOS pull-down network}
    \addcontentsline{toc}{subsubsection}{\hspace{7.5em}a. Tạo đồ thị mô hình hóa NMOS pull-down network}
    \item{Tìm đường đi Euler cho vùng NMOS pull-down network}
    \addcontentsline{toc}{subsubsection}{\hspace{7.5em}b. Tìm đường đi Euler cho vùng NMOS pull-down network}
\end{enumerate}
%2.1.3
\paragraph{Tạo đồ thị mô hình hóa PMOS pull-up network và tìm đường đi Euler}
%a. và b.
\begin{enumerate}[label=\alph*.]
    \item{Tạo đồ thị mô hình hóa NMOS pull-down network}
    \addcontentsline{toc}{subsubsection}{\hspace{7.5em}a. Tạo đồ thị mô hình hóa PMOS pull-down network}
    \item{Tìm đường đi Euler cho vùng NMOS pull-down network}
    \addcontentsline{toc}{subsubsection}{\hspace{7.5em}b. Tìm đường đi Euler cho vùng PMOS pull-down network}
\end{enumerate}
%2.1.4
\paragraph{Tìm điểm nối nguồn và output}
%2.1.5
\paragraph{Tổng kết mô hình hóa biểu thức Boolean sang Schematic Diagram}
%2.2
\subsubsection{Vẽ Stick Diagram trên Python}
%3.
\subsection{Kết quả mô phỏng}
Chèn hình ảnh stick diagram được vẽ ra và mô tả kết quả.
\newpage
%III.
\section{Tổng kết}
Từ bài tập lớn chuyển đổi biểu thức Boolean thành Stick Diagram, nhóm đã đạt một số yêu cầu sau:

• Biết cách vẽ Stick Diagram của biểu thức Boolean bất kì

• Tìm đường đi Euler cho vùng NMOS, CMOS

• Tìm được điểm nối nguồn và output của từng vùng NMOS, PMOS

• Sử dụng ngôn ngữ C++ và Python để tạo nên Stick Diagram.\\
Tuy nhiên, chương trình của nhóm vẫn còn cần phải cải thiện ở một số điểm như sau: 

• Chưa có khả năng tối ưu biểu thức đầu vào

• Số lượng phần tử logic quá lớn, có thể dẫn đến sai sót

• Việc vẽ các đường như VDD, GND, ndiff và pdiff chỉ là tương đối (phù hợp cho biểu thức 3-6 biến)
\end{document}