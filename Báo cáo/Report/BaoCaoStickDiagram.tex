\documentclass[a4paper,12pt]{article}
\usepackage[utf8]{inputenc}
\usepackage[vietnamese]{babel}
\usepackage{titlesec}
\usepackage{geometry}
\usepackage{graphicx}
\usepackage{fancyhdr}
\usepackage{tocloft}
\usepackage{hyperref}
\usepackage{setspace}
\usepackage{amsmath}
\usepackage{float}
\usepackage[most]{tcolorbox}
\usepackage{tikz}
\usepackage{everypage-1x}
\usepackage{atbegshi}
\usepackage{eso-pic}
\usepackage{enumitem}

\geometry{left=2cm,right=1.5cm,top=2cm,bottom=2.5cm}
\onehalfspacing

\setcounter{secnumdepth}{4}
\setcounter{tocdepth}{4}
\renewcommand{\theparagraph}{\thesubsubsection.\arabic{paragraph}}
\titleformat{\paragraph}[block]{\normalfont\normalsize\bfseries}{\theparagraph}{1em}{}
\titleformat{\section}{\bfseries\large}{\thesection}{1em}{}
\titleformat{\subsection}{\bfseries\normalsize}{\thesubsection}{1em}{}
\renewcommand{\cftsecleader}{\cftdotfill{\cftdotsep}}
\renewcommand{\cftsecfont}{\normalfont}
\renewcommand{\cftsecpagefont}{\normalfont}
\renewcommand{\cfttoctitlefont}{\normalfont\bfseries\Large}
\renewcommand{\cftaftertoctitle}{\hfill}
\renewcommand{\thesection}{\Roman{section}} % Section: La Mã
\renewcommand{\thesubsection}{\arabic{subsection}} % Subsection: 1, 2, 3...
\renewcommand{\thesubsubsection}{\arabic{subsection}.\arabic{subsubsection}} % Subsubsection: 1.1, 2.1,...
% Vẽ khung cho trang bìa
\AddToShipoutPictureBG*{%
    \AtPageLowerLeft{%
        \begin{tikzpicture}[remember picture, overlay]
            \draw[thick, black] 
                ([xshift=1.5cm, yshift=-1.5cm] current page.north west)
                rectangle 
                ([xshift=-1.5cm, yshift=2cm] current page.south east);
        \end{tikzpicture}
    }
}


\begin{document}
%----------------------------------
% Trang bìa
%----------------------------------
\thispagestyle{empty}
\begin{center}
\textbf{\Large ĐẠI HỌC QUỐC GIA THÀNH PHỐ HỒ CHÍ MINH}\\
\textbf{\Large TRƯỜNG ĐẠI HỌC CÔNG NGHỆ THÔNG TIN}\\
\textbf{\Large KHOA KỸ THUẬT MÁY TÍNH}\\[1cm]

\begin{figure}[H]
    \centering
    \includegraphics[width=0.3\textwidth]{../PNG/CE.png}
    \label{fig:LOGO_CE}\\
\end{figure}

\vspace {1cm}

\textbf{\Large BÁO CÁO ĐỒ ÁN THIẾT KẾ VI MẠCH SỐ}\\[0.5cm]
\textbf{\Large STICK DIAGRAM}\\[5cm]

\begin{flushleft}

\textbf{Lớp:} CE222.P21\\
\textbf{Giảng viên hướng dẫn:} ThS. Ngô Hiếu Trường\\
\textbf{Sinh viên thực hiện:} \\
- Đàm Vĩnh Khang : 22520606 \\
- Nguyễn Tuấn Khoa : 22520681\\[3.5cm]

\end{flushleft}
\end{center}
\begin{center}
\textbf{Thành phố Hồ Chí Minh, tháng 5 năm 2025}
\end{center}
\newpage
\setcounter{page}{1}
%----------------------------------
% Mục lục
%----------------------------------
\tableofcontents
\newpage

%----------------------------------
% Nội dung
%----------------------------------

%I.
\section{Tổng quát}
Viết phần giới thiệu ở đây.

%II.
\section{Quy trình thực hiện}

%1.
\subsection{Tìm hiểu thuật toán}
Chi tiết cho phần phân tích, ví dụ: mô hình hóa schematic diagram, tìm đường đi Euler/Hamilton, mô tả cách sử dụng graph.
%2.
\subsection{Hiện thực bằng Python và C++}
Viết mô tả cách cài đặt bằng Python, sử dụng thư viện matplotlib, networkx,... để vẽ đồ thị, stick diagram, các đoạn code chính.
%2.1
\subsubsection{Vẽ Schematic và tìm đường đi Euler trên C++}
%2.1.1
\paragraph{Hướng tiếp cận}
%2.1.2
\paragraph{Tạo đồ thị mô hình hóa NMOS pull-down network và tìm đường đi Euler}
%a. và b.
\begin{enumerate}[label=\alph*.]
    \item{Tạo đồ thị mô hình hóa NMOS pull-down network}
    \addcontentsline{toc}{subsubsection}{\hspace{6em}a. Tạo đồ thị mô hình hóa NMOS pull-down network}
    \item{Tìm đường đi Euler cho vùng NMOS pull-down network}
    \addcontentsline{toc}{subsubsection}{\hspace{6em}b. Tìm đường đi Euler cho vùng NMOS pull-down network}
\end{enumerate}
%2.1.3
\paragraph{Tạo đồ thị mô hình hóa PMOS pull-up network và tìm đường đi Euler}
%a. và b.
\begin{enumerate}[label=\alph*.]
    \item{Tạo đồ thị mô hình hóa NMOS pull-down network}
    \addcontentsline{toc}{subsubsection}{\hspace{6em}a. Tạo đồ thị mô hình hóa NMOS pull-down network}
    \item{Tìm đường đi Euler cho vùng NMOS pull-down network}
    \addcontentsline{toc}{subsubsection}{\hspace{6em}b. Tìm đường đi Euler cho vùng NMOS pull-down network}
\end{enumerate}
%2.1.4
\paragraph{Tìm điểm nối nguồn và output}
%2.1.5
\paragraph{Tổng kết mô hình hóa biểu thức Boolean sang Schematic Diagram}
%2.2
\subsubsection{Vẽ Stick Diagram trên Python}
%3.
\subsection{Kết quả mô phỏng}
Chèn hình ảnh stick diagram được vẽ ra và mô tả kết quả.
%III.
\section{Tổng kết}
Viết phần kết luận, tổng hợp những gì đạt được và những điểm còn hạn chế.
\end{document}